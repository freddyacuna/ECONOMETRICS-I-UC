CURSO: TÓPICOS EN ECONOMETRÍA
SIGLA: EAE252A
CRÉDITOS: 10
MÓDULOS: 02
REQUISITOS: EAE210B y EAE220B y EAE250A
CARÁCTER: OPTATIVO DE PROFUNDIZACIÓN
DISCIPLINA: ECONOMÍA


I. DESCRIPCIÓN

El curso está orientado a alumnos de pregrado que hayan cursado Macroeconomía I, Microeconomía I y Econometría, y tienen el objetivo de proveer herramientas 
de econometría y métodos numéricos que permitan aplicar los conocimientos teóricos adquiridos a lo largo de la rama de economia de la carrera.


II. OBJETIVOS

1. Programar códigos para rutinas de uso cotidiano no disponiblesen paquetes estásticos comerciale (v.g. E-views, Stata).

2. Aplicar conceptos elementales de econometría a través de la programación de apliaciones.

3. Ientificar ventajes y desventajas de metodologías alternativas para responder una pregunta económica.

4. Resolver problemas de muestra pequeña por meio de técnicas de remuestreo.

5. Abordar problemas de optiización no lineales en general.


III. CONTENIDOS

1. Programación en MATLAB.


2. Revisión de tópicos de econometría elemental.

2.1 Revisión de MCO: estimación e inferencia.

2.2 Violación de supuestos: heterocedasticidad/autocorrelación.

2.3 Problema con los datos: multicolinealidad /muestra reducida.

2.4 Ecuaciones simultánea, SUR.


3. Métodos de remuestreo.

3.1 Bootstrap.

3.2 Jacknife.

3.3 Cross validation.

3.4 Subsamplñing.

3.5 Aplicación: Corrección por sesgo de estadísticos clásicos.


4. Estimación no-paramétrica de densidades.

4.1 Motivación.

4.2 Funciones kernel.

4.3 Elección del ancho de banda h: trade-off sesgo/varianza.

4.4 Aplicación: Estimación de la densidad de ingresos familiares.


5. Métodos numéricos para solucionar ecuacioness no-lineales.

5.1 Motivación.

5.2 Forma funcional y aproximaciones.

5.3 Optimización unidimensional y multidmensional.

5.4 Aplicación: teoría del ingreso permanente.


6. Máxima verosimilitud.

6.1 Introducción.

6.2 Estimación.

6.3 Inferencia.

6.4 Cuasi-máxima verosimilitud/MCO.

6.5 Aplicación: Probabilidad de entrar al mercado laboral.


7. Método generalizado de momentos.

7.1 Introducción.

7.2 Estimación.

7.3 Inferencia.

7.4 Aplicación: estimación de la aversión al riesgo en un modelo microfundado.


IV. METODOLOGÍA

- Sesiones expositivas.
- Sesiones participativas.
- Uso de sala PC.


v. EVALUACIÓN

- Controles: 6
- Tareas: 3
- Trabajo final de curso.
- Discusión y participación.


VI. BIBLIOGRAFÍA

Mínima:

Alastair, H. Generalized method of moments. Advanced texts in econometrics. Oxford University Press, 2005.

Davidson, A. y D. Hinkley. Bootstrap Methods and their Application. Cambridge University Press, 2007.

Efron y Tibshirani. An introduction to the Bootstrap. Chapman & Hall/CRC., 1993.

Hansen, Bruce. Econometrics. University of Wisconsin, 2000. Manuscript.

Heer, B y A. Maussner. Dynamic General Equilibrium Modelling. Springer, Berlin, Germant, 2005.

Judo, k. Nunerical Methods in Economics. The MIT Press, 1998.

Matyas, L. Generalized Nethod of Moments Estimation. Cambridge University Press, 2003.

Silverman. Densisty Estimados for Statistcs and Analysis. Chapment & Hall, 1986.



PONTIFICIA UNIVERSIDAD CATÓLICA DE CHILE*
